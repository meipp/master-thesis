\chapter{Conclusion} \label{ch_conclusion}

We presented an algorithm to solve regex-constrained as well as unconstrained word equations and provided a reference implementation. We compiled and published three benchmark sets specialized for Nielsen-based approaches. Our implementation outperforms current state-of-the-art solvers on quadratic, regex-constrained word equations. We conclude that our approach is best applied as a tactic.

While this thesis provided an initial proof of concept showing the viability of our approach, much future work can still be done.

Firstly, our solver supports neither length constraints nor equation systems containing more than one equation. This means we are not compatible with the entire SMT-LIB standard and thus cannot compete in string solving competitions.

Secondly, this solver is currently implemented as a standalone solver. As explained earlier, it would make most sense to implement it as a solver tactic for either one of the state-of-the-art solvers or in a portfolio solver.

Lastly, one could extend the algorithm in multiple ways.
For once, we only implemented Brzozowski derivatives. One could instead implement transition regexes to see whether their claimed advantage \cite{transition-regex} also holds within our approach. Furthermore, we noted that Dijkstra's algorithm can be used when given a heuristic. One could try different heuristics ranking word equations in terms of prospected solvability.
Finally, we note that simple length constraints can also be expressed as derivative-based constraints. A future version of our solver could therefore possibly also handle length constraints.
One last drawback of our algorithm is that it is not guaranteed to terminate on some unsatisfiable equations. This problem can be solved by making use of regex quotients.